\begin{longtable}[H]{ | l | p{10cm} | }
\hline
{\cellcolor[rgb]{0.753,0.753,0.753}}\textbf{ID}  & 1.8 \\ \hline
{\cellcolor[rgb]{0.753,0.753,0.753}}\textbf{Name} & Farmer participates to a thread in a forum \\ \hline
{\cellcolor[rgb]{0.753,0.753,0.753}}\textbf{Actors} & Farmer \\ \hline
{\cellcolor[rgb]{0.753,0.753,0.753}}\textbf{Entry conditions} & The farmer is already registered in the system \\ \hline
{\cellcolor[rgb]{0.753,0.753,0.753}}\textbf{Input} & A message sent by the farmer\\ \hline
{\cellcolor[rgb]{0.753,0.753,0.753}}\textbf{Event flows} &
\begin{itemize}
    \item The farmer opens the application
    \item The farmer clicks on the "Forum" button
    \item The system display the screen relative to the forum, and in particular the list of all the thread order by the most recent to the least recent 
    \item The farmer selects a thread
    \item The system displays all the messages belonging to the thread and the form used to append new messages to the thread itself
    \item The farmer appends a message to the thread 
    \item The farmer clicks on the "Comment" button
    \item The system appends the new message to the thread
\end{itemize}
\\ \hline
{\cellcolor[rgb]{0.753,0.753,0.753}}\textbf{Exit Condition} & The farmer successfully publish their message into the thread\\ \hline
{\cellcolor[rgb]{0.753,0.753,0.753}}\textbf{Output} & 
The thread has a new message in it
\\ \hline
{\cellcolor[rgb]{0.753,0.753,0.753}}\textbf{Exceptions} & The Farmer closes the application before clicking the "Comment" button. By doing so all the content is lost.
\\ \hline
\caption{Use case 1.8 - Farmer participates to a thread in a forum}
\\
\end{longtable}